There will be a single primary view for the user interface. The main concept of the user interface is to allow the user to interact with and affect the instance of the TA/Professor/Course matching. This view will contain a representation of the current TA/Professor/Course matching and will allow the user to view the triples in the current instance of the maximal matching. The exact visual representation of the matching will be determined as the project progresses. Several options for interactive visualization are being considered. A graph representation of the matching (where the nodes represent TA's, courses, or professors) may be the most efficient method of providing a visual representation of the entity relationships. The group may consider adapting a hierarchical graph or possibly a radial graph representing all possible sets of TA/Professor/Course triples where the edges between nodes represent possible triples in the maximum matching. The user will be able to interact with the graph by clicking (or hovering) over a node to see the result of the matching. The user will also be able to interact with the current instance of the matching by assigning (or un-assigning) triples. User feedback will trigger the algorithm to recalculate the maximal matching under the user defined assignments and the corresponding visualization will be displayed.  
