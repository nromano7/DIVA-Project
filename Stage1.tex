The goal for this project by the end of the semester is to have one administrative view that incorporates a visual interface for the delivery of user feedback to the 3D-Matching algorithm, and a visual representation of the final matching that is output by the algorithm. 

An example scenario would be as follows:
\\
The CS department administrator has collected the list of TA applications in the CS department and is ready to match them with professors and courses. The administrator logs onto RTAS and runs the 3D-Matching algorithm. The algorithm returns a matching of size n, as well as a visualization of the list of accepted applications. The administrator then notices that Professor A submit a request to have Student A as his TA for Class A. The administrator goes through the preference selection interface and selects the tuple (Student A, Professor A, Course A). This adds the tuple to the algorithm's priority list. He re-runs the algorithm, which returns a new matching with the same size as the old one. The administrator decides to keep it. The administrator inputs a second request: (Student B, Professor B, Course B). The algorithm runs and returns a matching that is smaller than the old matching. The administrator then declines that request and removes tuple (Student B, Professor B, Course B) from the priority list. The administrator then finalizes the final matching and assigns the TAs to their respective professors and courses. 

A tentative timeline for project completion is as follows:

\begin{itemize} 
	\item{\textbf{Feb 19:} A dummy database for use for this project will be generated and prepared for use.} 
	
	\item{\textbf{Mar 6:} The 3D-Matching approximation algorithm will be completed and ready for use.}
	
	\item{\textbf{Mar 20:} A maximal matching visualization prototype will be finished.}
	
	\item{\textbf{Apr 3:} The matching visualization will be completed. The user feedback interactive interface will be completed.}
	
	\item{\textbf{Apr 17:} The administrative system view will be completed.}
	
	\item{\textbf{Apr 24:} The powerpoint presentation will be completed.}
	
	\item{\textbf{Apr 30:} The project report will be completed.}
	
\end{itemize}
