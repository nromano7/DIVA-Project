\documentclass[10pt,conference]{IEEEtran}

\ifCLASSINFOpdf
	\usepackage[pdftex]{graphicx}
	%\graphicspath{{./figs/}}
	\DeclareGraphicsExtensions{.pdf,.jpeg,.png}
\else
	\usepackage[dvips]{graphicx}
	%\graphicspath{{./figs/}}
	\DeclareGraphicsExtensions{.eps}
\fi

\usepackage[cmex10]{amsmath}
\usepackage[tight,footnotesize]{subfigure}
\usepackage{xcolor}
\usepackage[lined,ruled]{algorithm2e}

\usepackage[latin1]{inputenc}
\usepackage{tikz}
\usetikzlibrary{shapes}
\usetikzlibrary{arrows}

\usepackage[]{algorithm2e}

\newtheorem{property}{Property}
\newtheorem{proposition}{Proposition}
\newtheorem{theorem}{Theorem}
\newtheorem{conjecture}{Conjecture}
\newtheorem{question}{Question}
\newtheorem{definition}{Definition}
\newtheorem{corollary}{Corollary}

\makeatletter
\pgfdeclareshape{datastore}{
\inheritsavedanchors[from=rectangle]
\inheritanchorborder[from=rectangle]
\inheritanchor[from=rectangle]{center}
\inheritanchor[from=rectangle]{base}
\inheritanchor[from=rectangle]{north}
\inheritanchor[from=rectangle]{north east}
\inheritanchor[from=rectangle]{east}
\inheritanchor[from=rectangle]{south east}
\inheritanchor[from=rectangle]{south}
\inheritanchor[from=rectangle]{south west}
\inheritanchor[from=rectangle]{west}
\inheritanchor[from=rectangle]{north west}
\backgroundpath{
    %  store lower right in xa/ya and upper right in xb/yb
\southwest \pgf@xa=\pgf@x \pgf@ya=\pgf@y
\northeast \pgf@xb=\pgf@x \pgf@yb=\pgf@y
\pgfpathmoveto{\pgfpoint{\pgf@xa}{\pgf@ya}}
\pgfpathlineto{\pgfpoint{\pgf@xb}{\pgf@ya}}
\pgfpathmoveto{\pgfpoint{\pgf@xa}{\pgf@yb}}
\pgfpathlineto{\pgfpoint{\pgf@xb}{\pgf@yb}}
 }
}
\makeatother

\newcommand{\riham}[1]{{\color{red}{#1}}}
\newcommand{\james}[1]{{\color{blue}{#1}}}


\begin{document}

\title{CS526 GCRS - Graduate Course Request System - Spring 2018}
\author{
\IEEEauthorblockN{Wael Ayadi, Nick Romano}
\IEEEauthorblockA{Rutgers University\\ Piscataway, NJ, USA\\
Email: wael.ayadi@rutgers.edu, nick.p.romano@gmail.com}
%\and
%\IEEEauthorblockN{Alan Turing}
%\IEEEauthorblockA{Rutgers University\\
% Piscataway, NJ, USA\\
% Email: alan1936@cs.rutgers.edu}
%\and
%%\IEEEauthorblockN{Third group member}
%\IEEEauthorblockA{Rutgers University\\
%\Piscataway, NJ, USA\\
%Email: Third_group_member@scarletmail.rutgers.edu}
}

\maketitle
\begin{abstract}
\textnormal{
We will create a Graduate Course Request System that will allow for easier and faster delivery and approval of graduate course requests for undergraduate students. It will incorporate our multi-purpose Visual Transcript visualization, which provides an interactive interface for extracting useful information from a student's course history. 
}
\end{abstract}
%\onecolumn \maketitle \normalsize \vfill

\IEEEpeerreviewmaketitle
%%%%%%%%%%%%%%%%%%%%%%%%%%%%%%%%%%%%%%%%%%%%%%%%%%%%%%%%%%%%%%%%%%%%%%%%%%%%%%%%%%%%%%%%%%%%%%%%%%%%%%%%%
\section{Project Description}\label{sec:1. Project Description}
%%%%%%%%%%%%%%%%%%%%%%%%%%%%%%%%%%%%%%%%%%%%%%%%%%%%%%%%%%%%%%%%%%%%%%%%%%%%%%%%%%%%%%%%%%%%%%%%%%%%%%%%%
\textnormal{
The Rutgers University Graduate CS Department allows for undergraduate students to register for graduate courses through a special request process. This process, as it is now, involves manually signing a form for every graduate course that the student wishes to register for. This signed form must then be signed by the instructor of the course that is being requested. Afterwards, the form is given to the graduate secretary of the Graduate CS Department. Depending on whether or not the requested course requires a prerequisite override (it often does, considering the student submitting this request is an undergraduate student), the form would then have to be forwarded to the graduate school for approval. This step of the process specifically may take up to a full week to complete! Once the course request is finally approved by the graduate school, the student receives his Special Permission Number (SPN) for the requested course.
}

Not only is this process tedious to describe, but it is tedious in practice as well. The student often has to chase down instructors for their approvals in person, as it is not realistic for them to constantly check their emails for course requests, as well as do the necessary background checks on the student to see if an approval for their request is appropriate. Further complications occur if an instructor is on sabbatical leave, if the graduate secretary is taking a day off, etc.. To make matters worse, students are bound by the add/drop week deadline for course registration at the beginning of every semester, thus making this registration process a timely matter that cannot always be done before the start of the semester. These issues are further compounded by the fact that the Rutgers Graduate CS Department has been growing in terms of student enrollment. The increased frequency of these requests end up delaying this process even further. 

We propose the GCRS - Graduate Course Request System, which is a system that can be linked with Rutgers's Central Authentication System (CAS) to allow for Rutgers faculty to digitally manage and pass along student graduate course requests. This system will use our Visual Transcript visualization, which will allow easy and fast data visualization of a student's course history and related data. Incorporating Visual Transcript into GCRS will give an interactive interface for faculty to quickly see the information they require when it comes to approving these requests. 

%\subsection{Stage1 - The Requirement Gathering Stage. } \label{sec:1	Requirement Gathering Stage. } 
%\textnormal{
Get a realistic project idea that includes potential real world scenarios,
with a description of the different user types along with their interactions with the system 
as well as the system feedback to them, according to their information needs. 
This stage also requires the specification of the different constraints and restrictions
that need to be enforced depending on the different types of user (system interactions). 
The deliverables for this stage include the following items:
} 

\begin{itemize} 
\item{A general description (in plain English) of your project's deliverables (understandable by computer illiterate users).} 

\item{ A specific description of at least three types of users. }
	
\item{ A description of detailed real world scenarios (at least 2 scenarios) representing those typical interactions between the different user types and the system (including inputs and outputs and data types).}
	

\item{A description of detailed real world scenarios (at least 2 scenarios) representing those typical interactions between the different user types and the system (including inputs and outputs and data types). }

\item{A detailed time line for completion of the major implementaion stages together with the division of labor including testing, documentation, evaluaton, project report, and power point presentation.}

\end{itemize}

Please insert your deliverables for Stage1 as follows:

\begin{itemize} 
\item{The general system description: } 
Please insert the system description here.
\item{The three types of users (grouped by their data access/update rights): }
Please insert the users types in here, as follows:
\item{The user's interaction modes: }
Please insert the user's interaction modes here.
\item{The real world scenarios: }
Please insert the real world scenarios in here, as follows.
	\begin{itemize} 
	\item{Scenario1 description: }
	Please insert Scenario1 description here.
	\item{System Data Input for Scenario1: }
	Please insert System Data Input for Scenario1 here.
	\item{Input Data Types for Scenario1: }
	Please insert Input Data Types for Scenario1 here.
	\item{System Data Output for Scenario1: }
	Please insert System Data Output for Scenario1 here.
	\item{Output Data Types for Scenario1: }
	Please insert Output Data Types for Scenario1 in here.
	\item {Please repeat that pattern for each scenario (at least 2 scenarios per user).}
	\end{itemize}
Please repeat that pattern for each user type.
\item{Project Time line and Divison of Labor.}
Please insert here the time line and the corresponding implementation tasks.
\end{itemize}

\subsection{Stage1 - The Requirement Gathering Stage. }\label{sec:1 Requirement Gathering Stage. }
%%%%%%%
\textnormal{
Get a realistic project idea that includes potential real world scenarios,
with a description of the different user types along with their interactions with the system 
as well as the system feedback to them, according to their information needs. 
This stage also requires the specification of the different constraints and restrictions
that need to be enforced depending on the different types of user (system interactions). 
The deliverables for this stage include the following items:
} 

\begin{itemize} 
\item{A general description (in plain English) of your project's deliverables (understandable by computer illiterate users).} 

\item{ A specific description of at least three types of users. }
	
\item{ A description of detailed real world scenarios (at least 2 scenarios) representing those typical interactions between the different user types and the system (including inputs and outputs and data types).}
	

\item{A description of detailed real world scenarios (at least 2 scenarios) representing those typical interactions between the different user types and the system (including inputs and outputs and data types). }

\item{A detailed time line for completion of the major implementaion stages together with the division of labor including testing, documentation, evaluaton, project report, and power point presentation.}

\end{itemize}

Please insert your deliverables for Stage1 as follows:

\begin{itemize} 
\item{The general system description: } 
Please insert the system description here.
\item{The three types of users (grouped by their data access/update rights): }
Please insert the users types in here, as follows:
\item{The user's interaction modes: }
Please insert the user's interaction modes here.
\item{The real world scenarios: }
Please insert the real world scenarios in here, as follows.
	\begin{itemize} 
	\item{Scenario1 description: }
	Please insert Scenario1 description here.
	\item{System Data Input for Scenario1: }
	Please insert System Data Input for Scenario1 here.
	\item{Input Data Types for Scenario1: }
	Please insert Input Data Types for Scenario1 here.
	\item{System Data Output for Scenario1: }
	Please insert System Data Output for Scenario1 here.
	\item{Output Data Types for Scenario1: }
	Please insert Output Data Types for Scenario1 in here.
	\item {Please repeat that pattern for each scenario (at least 2 scenarios per user).}
	\end{itemize}
Please repeat that pattern for each user type.
\item{Project Time line and Divison of Labor.}
Please insert here the time line and the corresponding implementation tasks.
\end{itemize}


\subsection{Stage2 - The Design Stage. }\label{sec: 2:The Design Stage.}
%%%%%%%%%%%%%%%%%%%%%%%%%%%%%%%%%%%%%%%%%%%%%%%%%%%%%%%%%%%%%%%%%%%%%%%%%%%%%%%%%%%%%%%%%%%%%%%%%%%%%%%%%%
\textnormal{
Transform the project requirements into a system flow diagram, specifyng the different algorithms, data types and structures required for processing and their associated operations.  
The deliverables for this stage include the system flow diagram containing a graphical representation and  textual descriptions of the corresponding data trasnformations, high level pseudo code of the overall system operation, and overall system time and space complexity.}

%\begin{itemize} 
%\item{ }
%A brief textual description of the overall flow diagram (along with its functional operation in the different user scenarios described in the first stage of the project).
%\item{ }
%A specification of each algorithm and associated data structures together with its entities, attributes, and operations ( include an English description of how they relate to your user scenario(s)).

%\end{itemize}
Please insert your deliverables for Stage2 as follows:
\begin{itemize} 
\item{  Short Textual Project Description. }
Please insert here the flow diagram textual description here together with its overall time and space complexity.
\item{ Flow Diagram. }
Please insert your system Flow Diagram here.
\item{ High Level Pseudo Code System Description. }
Please insert high level pseudo-code describing the major system modules as per your flow diagram.
\item{Algorithms and  Data Structures. }
Please insert a brief description of each major Algorithm and its associated data structures here.
\end{itemize}

\begin{itemize} 
\item{  Flow Diagram Major Constraints.}
Please insert here the integrity constraints:
\begin{itemize} 
\item{ Integrity Constraint. }
Please insert the first integrity constraint in here together with its description and justification. }
\end{itemize}
Please repeat the pattern for each integrity constraint.
\end{itemize}
}


\subsection{Stage3 - The Implementation Stage. }\label{sec: 3 The Implementation Stage.}
%%%%%%%%%%%%%%%%%%%%%%%%%%%%%%%%%%%%%%%%%%%%%%%%%%%%%%%%%%%%%%%%%%%%%%%%%%%%%%%%%%%%%%%%%


\textnormal{
Specify the language and programming environemnt you used for your implementation.
%Building the corresponding relational tables, according to the proposed ER model described in the previous phase %enforcing the different integrity constraints.  
The deliverables for this stage include the following items:
\begin{itemize} 
\item{}
Sample small data snippet. 
%The SQL tables that represent the ER project model, along with at least 3-5 rows of concrete data per table.
\item{}
Sample small output
%The normalization steps for each table, along with explanations/justifications of each normalization step.
\item{}
Working code
%The SQL table after the normalization steps (showing all table attributes).
\item{}
Demo and sample findings
%The SQL statements used to create the SQL tables, including the required triggers as well as the integrity constraints. At %least 2 triggers and 2 of each of the following constraint types have to exist in the project tables overall: 
\begin{itemize} 
\item{}
	Data size: In terms of  RAM size;  Disk Resident?; Streaming ?;  
\item{}
	List the most interestng findings in the data if it is a Data Exploration Project. For other project types consult with your project supervisor what the corresponding outcomes shall be. Concentrate on demonstrating the Usefuness and Novelty of your application.
%Whether some users will be denied access and/or updates to some data according to their roles (for example: student1 %can not access other students' ' grades, so a violation error pops up upon that action. Another example: a sales person %can see an item price, but can not change it, since only a manger can, also a violation error pops up upon that update %attempt).
\end{itemize}
\end{itemize}
}


\subsection{Stage4 -	User Interface. }\label{sec: 4. User Interface.}
%%%%%%%%%%%%%%%%%%%%%%%%%%%%%%%%%%%%%%%%%%%%%%%%%%%%%%%%%%%%%%%%%%%%%%%%%%%%%%%%%%%%%%%%%%%%%%%%%%%%%%%%%%
There will be two main components of the user interface: the requests view (i.e. the main page), and the visual transcript view. In addition there will be a login page for the entire system. The type of user will determine exactly what content and/or interactivity is available in each view. The user type will be assessed via login credentials submitted on the login page. For example, the requests main page will give a student the option to submit a request. This page will also display any open requests and additionally it may show any closed requests that were previously completed. On the opposite side of the pipeline, faculty/admin will have a similar view showing all of the requests that require completion. Both types of users will have access to the visual transcript view which will provide an interactive visualization of the student's course history.This view will allow student users to confirm whether they meet the necessary requirements, and will allow faculty/admin to quickly identify whether a student has the necessary coursework to needed to approve the request. The faculty/admin view will contain "Approve" and "Deny" buttons which will either continue or terminate the request, respectively. Each user view is described in further detail herein. As the login page will not be a main component of the user interface, it is assumed the user has already logged in.

For student users, the requests main page will contain a link that will take the student to their visual transcript view, and buttons to submit a new request. Any open requests will be visible in a list with options to view its status. Upon click of an open request, a window will open showing details on the status of that request. Alternatively, the status of an open request may be shown upon hover. Any closed requests (i.e. requests that were previously completed will be displayed at the bottom of the page). 

For faculty/admin users, the requests main page will appear much like the students requests page however the options available to the user will be different. In this view, faculty or admin will not require the option to submit a request. Instead all open requests (submitted by the students) will be displayed. Similar to the student view of the requests main page, the faculty/admin view will allow the user to view the status of any request and previously completed requests will be shown at the bottom of the page. The faculty/admin user will have the options to approve or deny the request (or multiple requests) from this page. The faculty/admin user will also have the option to access the students virtual transcript in order to view the students course history.

The purpose of this virtual transcript is to provide a visual and interactive aspect to the traditional course list and grade type transcript. This visual transcript will allow all users to quickly interpret a student's academic history. The data to be visualized in this virtual transcript will be the complete course history of the particular student as well as a visual representation of the students strengths (or weaknesses) It is imagined that the students course history will be displayed in sequential order through an adapted bar chart where the bars are located over each course and the scale of the vertical axis represents the grade the student received for the course. A color dimension will be added to represent the course level (e.g. 100, 200, 300-level courses). Additionally the students strengths and/or weaknesses in certain course concentrations (e.g. machine learning, security, data science, etc.) will be displayed via a set radial progress bars.


\textnormal{
Describe a User Interface (UI) to your application along with the related information that will be shown on each interface view (How users will query or navigate the data and view the query or navigation results). The emphasis should be placed on the process a user needs to follow in order to meet a particular information need in a user-friendly manner.
The deliverables for this stage include the following items :
}
\begin{itemize} 
\item{The modes of user interaction with the data (text queries, mouse hovering, and/or mouse clicks ?).} 
\item{The error messages that will pop-up when users access and/or updates are denied   }
\item{The information messages or results that wil pop-up in response to user interface events. }
	
\item{ The error messages in response to data range constraints violations.}
	
\item{ The interface mechanisms that activate different views in order to facilitate data accesses, according to users'  needs. }
	
\item{Each view created must be justified. Any triggers built upon those views should be explained and justified as well. At least one project view should be created with a justification for its use. }	
\end{itemize}

Please insert your deliverables for Stage4 as follows:
\begin{itemize} 
\item{The initial statement to activate your application with the corresponding initial UI screenshot}
	
\item{Two different  sample navigation user paths through the data exemplifying the different modes of interaction and the corresponding screenshots. }
\item{}
	The error messages popping-up when users access and/or updates are denied (along with explanations and examples):
	\begin{itemize} 
	\item{The error message: }
	\item{The error message explanation (upon which violation it takes place): }
	Please insert the error message explanation in here.
	\item{The error message example according to user(s) scenario(s): }
	Please insert the error message example in here.
	 \end{itemize}
\item{}
	The information messages or results that pop-up in response to user interface events.
	\begin{itemize} 
	\item{The information message: }
	Please insert the error message in here.
	\item{The information message explanation and the corresponding event trigger }
	\item{The error message example in response to data range constraints and the coresponding user's scenario }
	Please insert the error message example in here.
	 \end{itemize}
\item{}
	The  interface mechanisms that activate different views.
	\begin{itemize} 
	\item{The interface mechanism: }
	Please insert the interface mechanism here.
	 \end{itemize}
 
\end{itemize}


\section{Alternative Project Extension.}\label{sec:2. Alternative Project Extension.}
%%%%%%%%%%%%%%%%%%%%%%%%%%%%%%%%%%%%%%%%%%%%%%%%%%%%%%%%%%%%%%%%%%%%%%%%%%%%%%%%%%%%%%%%%%%%%%%%%%%%%%%%%%
The project in its current state deals with two entity sets (students and faculty/admin). Per discussion with Professor Abello, in order to make the project a bit more interesting, we propose an alternative/extension to the project presented to this point. The alternative project will consider three entity sets instead of two. These entity sets are: TAs, Faculty, and Classes. The relationship between these entity sets represents a real world problem that the Department of Computer Science at Rutgers University faces every semester. That problem is matching potential TA's with the classes they wish to assist, and with faculty. This is essentially the classic problem of 3D Matching. The 3D Macthing problem is NP-Hard which makes it a difficult problem to solve algorithmically. This creates an opportunity to develop an interface for human-computer interaction in which a user aids the search algorithm in finding a solution in an efficient amount of time. This idea will be explored further and presented in future updates to this project proposal.


\bibliographystyle{IEEEtran}
%\bibliography{IEEEabrv,bib_queyroi_abello2013}
%\bibliography{bib_queyroi_abello2013}

\end{document}


