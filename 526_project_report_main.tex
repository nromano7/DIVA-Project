\documentclass[10pt,conference]{IEEEtran}

\ifCLASSINFOpdf
	\usepackage[pdftex]{graphicx}
	%\graphicspath{{./figs/}}
	\DeclareGraphicsExtensions{.pdf,.jpeg,.png}
\else
	\usepackage[dvips]{graphicx}
	%\graphicspath{{./figs/}}
	\DeclareGraphicsExtensions{.eps}
\fi

\usepackage[cmex10]{amsmath}
\usepackage[tight,footnotesize]{subfigure}
\usepackage{xcolor}
\usepackage[lined,ruled]{algorithm2e}

\usepackage[latin1]{inputenc}
\usepackage{tikz}
\usetikzlibrary{shapes}
\usetikzlibrary{arrows}

\usepackage[]{algorithm2e}

\newtheorem{property}{Property}
\newtheorem{proposition}{Proposition}
\newtheorem{theorem}{Theorem}
\newtheorem{conjecture}{Conjecture}
\newtheorem{question}{Question}
\newtheorem{definition}{Definition}
\newtheorem{corollary}{Corollary}

\makeatletter
\pgfdeclareshape{datastore}{
\inheritsavedanchors[from=rectangle]
\inheritanchorborder[from=rectangle]
\inheritanchor[from=rectangle]{center}
\inheritanchor[from=rectangle]{base}
\inheritanchor[from=rectangle]{north}
\inheritanchor[from=rectangle]{north east}
\inheritanchor[from=rectangle]{east}
\inheritanchor[from=rectangle]{south east}
\inheritanchor[from=rectangle]{south}
\inheritanchor[from=rectangle]{south west}
\inheritanchor[from=rectangle]{west}
\inheritanchor[from=rectangle]{north west}
\backgroundpath{
    %  store lower right in xa/ya and upper right in xb/yb
\southwest \pgf@xa=\pgf@x \pgf@ya=\pgf@y
\northeast \pgf@xb=\pgf@x \pgf@yb=\pgf@y
\pgfpathmoveto{\pgfpoint{\pgf@xa}{\pgf@ya}}
\pgfpathlineto{\pgfpoint{\pgf@xb}{\pgf@ya}}
\pgfpathmoveto{\pgfpoint{\pgf@xa}{\pgf@yb}}
\pgfpathlineto{\pgfpoint{\pgf@xb}{\pgf@yb}}
 }
}
\makeatother

\newcommand{\riham}[1]{{\color{red}{#1}}}
\newcommand{\james}[1]{{\color{blue}{#1}}}


\begin{document}

\title{The Rutgers Administrative Management System (RAM)}

\author{
\IEEEauthorblockN{Wael Ayadi, Nick Romano}
\IEEEauthorblockA{CS526 - Spring 2018 \\
	Rutgers University\\ 
	Piscataway, NJ, USA\\
	Email: wael.ayadi@rutgers.edu, nick.p.romano@gmail.com}
}

\maketitle
\begin{abstract}
	The Rutgers TA Allocation System (RTAS) is a web application that takes a list of Teaching Assistant (TA) applications as input, processes them using a 3D-Matching algorithm, and allows the user to provide feedback to the algorithm through human-computer interaction in order to extract a collection of approved TA applications that the user is satisfied with. Each TA application is a 3-tuple from the cross-product of three entity sets: the TAs submitting applications, professors, and courses. A generality of this use case is the well known 3D-Matching problem, which is an NP-hard problem where the goal is to find an optimal matching between TAs, professors, and courses. The RTAS will use an approximation algorithm to perform its processing, and it will provide a visual interface that will allow the user to provide feedback to the approximation algorithm in hopes of returning a better output.
\end{abstract}
%\onecolumn \maketitle \normalsize \vfill

\IEEEpeerreviewmaketitle
%%%%%%%%%%%%%%%%%%%%%%%%%%%%%%%%%%%%%%%%%%%%%%%%%%%%%%%%%%%%%%%%%%%%%%%%%%%%%%%%%%%%%%%%%%%%%%%%%%%%%%%%%
\section{Project Description}\label{sec:1. Project Description}
The Rutgers University Graduate CS Department allows for undergraduate students to register for graduate courses through a special request process. This process, as it is now, involves manually signing a form for every graduate course that the student wishes to register for. This signed form must then be signed by the instructor of the course that is being requested. Afterwards, the form is given to the graduate secretary of the Graduate CS Department. Depending on whether or not the requested course requires a prerequisite override (it often does, considering the student submitting this request is an undergraduate student), the form would then have to be forwarded to the graduate school for approval. This step of the process specifically may take up to a full week to complete! Once the course request is finally approved by the graduate school, the student receives his Special Permission Number (SPN) for the requested course.

Not only is this process tedious to describe, but it is tedious in practice as well. The student often has to chase down instructors for their approvals in person, as it is not realistic for them to constantly check their emails for course requests, as well as do the necessary background checks on the student to see if an approval for their request is appropriate. Further complications occur if an instructor is on sabbatical leave, if the graduate secretary is taking a day off, etc.. To make matters worse, students are bound by the add/drop week deadline for course registration at the beginning of every semester, thus making this registration process a timely matter that cannot always be done before the start of the semester. These issues are further compounded by the fact that the Rutgers Graduate CS Department has been growing in terms of student enrollment. The increased frequency of these requests end up delaying this process even further. 

We propose the GCRS - Graduate Course Request System, which is a system that can be linked with Rutgers's Central Authentication System (CAS) to allow for Rutgers faculty to digitally manage and pass along student graduate course requests. This system will use our Visual Transcript visualization, which will allow easy and fast data visualization of a student's course history and related data. Incorporating Visual Transcript into GCRS will give an interactive interface for faculty to quickly see the information they require when it comes to approving these requests. 


\end{document}


