\documentclass[10pt,conference]{IEEEtran}

\ifCLASSINFOpdf
	\usepackage[pdftex]{graphicx}
	%\graphicspath{{./figs/}}
	\DeclareGraphicsExtensions{.pdf,.jpeg,.png}
\else
	\usepackage[dvips]{graphicx}
	%\graphicspath{{./figs/}}
	\DeclareGraphicsExtensions{.eps}
\fi

\usepackage[cmex10]{amsmath}
\usepackage[tight,footnotesize]{subfigure}
\usepackage{xcolor}
\usepackage[lined,ruled]{algorithm2e}

\usepackage[latin1]{inputenc}
\usepackage{tikz}
\usetikzlibrary{shapes}
\usetikzlibrary{arrows}

\usepackage[]{algorithm2e}

\newtheorem{property}{Property}
\newtheorem{proposition}{Proposition}
\newtheorem{theorem}{Theorem}
\newtheorem{conjecture}{Conjecture}
\newtheorem{question}{Question}
\newtheorem{definition}{Definition}
\newtheorem{corollary}{Corollary}

\makeatletter
\pgfdeclareshape{datastore}{
\inheritsavedanchors[from=rectangle]
\inheritanchorborder[from=rectangle]
\inheritanchor[from=rectangle]{center}
\inheritanchor[from=rectangle]{base}
\inheritanchor[from=rectangle]{north}
\inheritanchor[from=rectangle]{north east}
\inheritanchor[from=rectangle]{east}
\inheritanchor[from=rectangle]{south east}
\inheritanchor[from=rectangle]{south}
\inheritanchor[from=rectangle]{south west}
\inheritanchor[from=rectangle]{west}
\inheritanchor[from=rectangle]{north west}
\backgroundpath{
    %  store lower right in xa/ya and upper right in xb/yb
\southwest \pgf@xa=\pgf@x \pgf@ya=\pgf@y
\northeast \pgf@xb=\pgf@x \pgf@yb=\pgf@y
\pgfpathmoveto{\pgfpoint{\pgf@xa}{\pgf@ya}}
\pgfpathlineto{\pgfpoint{\pgf@xb}{\pgf@ya}}
\pgfpathmoveto{\pgfpoint{\pgf@xa}{\pgf@yb}}
\pgfpathlineto{\pgfpoint{\pgf@xb}{\pgf@yb}}
 }
}
\makeatother

\newcommand{\riham}[1]{{\color{red}{#1}}}
\newcommand{\james}[1]{{\color{blue}{#1}}}


\begin{document}

\title{The Rutgers Administrative Management System (RAM)}

\author{
\IEEEauthorblockN{Wael Ayadi, Nick Romano}
\IEEEauthorblockA{CS526 - Spring 2018 \\
	Rutgers University\\ 
	Piscataway, NJ, USA\\
	Email: wael.ayadi@rutgers.edu, nick.p.romano@gmail.com}
}

\maketitle
\begin{abstract}
	The Rutgers TA Allocation System (RTAS) is a web application that takes a list of Teaching Assistant (TA) applications as input, processes them using a 3D-Matching algorithm, and allows the user to provide feedback to the algorithm through human-computer interaction in order to extract a collection of approved TA applications that the user is satisfied with. Each TA application is a 3-tuple from the cross-product of three entity sets: the TAs submitting applications, professors, and courses. A generality of this use case is the well known 3D-Matching problem, which is an NP-hard problem where the goal is to find an optimal matching between TAs, professors, and courses. The RTAS will use an approximation algorithm to perform its processing, and it will provide a visual interface that will allow the user to provide feedback to the approximation algorithm in hopes of returning a better output.
\end{abstract}
%\onecolumn \maketitle \normalsize \vfill

\IEEEpeerreviewmaketitle
%%%%%%%%%%%%%%%%%%%%%%%%%%%%%%%%%%%%%%%%%%%%%%%%%%%%%%%%%%%%%%%%%%%%%%%%%%%%%%%%%%%%%%%%%%%%%%%%%%%%%%%%%
\section{Project Description}\label{sec:1. Project Description}
The a great number of Rutgers departments allow for graduate students to apply to become Teaching Assistants (TA), where they will be paired up with a professor from that respective department who will act as their mentor, and they will be assigned a course from that respective department that they will be helping to teach. Each TA can only be assigned to one professor and one course, and every professor wants to have a TA that will assist him in his research. Therefore, we would like the maximum number of TA applications approved. However, this is not easy to guarantee, since there can be conflicting TA applications (e.g. two TAs can apply to work under the same professor). As a result, this TA assignment process often tends to be very difficult and tedious. This is because this problem is an example of the 3D-Matching Problem, which one of the well-known NP-hard problems in computational complexity theory. 

This is a 3D-Matching problem, where the three entity sets that are being matched are TAs, professors, and courses. The goal is to find the largest \textbf{matching}, where a matching is a collection of TA applications that do not conflict with each other. Because this problem is NP-hard, the fastest known algorithm for finding a maximum-sized matching has an exponential run-time, which means that it is too slow to be feasible to use in practice. There are approximation algorithms that we can use to derive solutions for the 3D-Matching Problem that are at least a fraction as good as the optimal solution. 

However, the approximation algorithm can output solutions that are quite varied, and normally do not have the means to accommodate for a user's preferences in the matchings. For example, a professor may have had a good experience with a certain TA in the past and prefers that specific TA to work with him over other TAs. However, forcing his preference onto the final matching may result in a final matching that has a smaller size than a final matching without his preference enforced, which can have negative consequences on other professors or courses. 

We propose the Rutgers TA Assignment System (RTAS), which a system that will provide the means to incorporate human-computer interaction into this TA application approval process. The system will provide a visual interface, which will allow for user preferences to be entered. Afterwards, the system will display how the entered user preferences change the quality of the final matching. 

\end{document}


