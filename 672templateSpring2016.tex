\documentclass[10pt,conference]{IEEEtran}

\ifCLASSINFOpdf
	\usepackage[pdftex]{graphicx}
	%\graphicspath{{./figs/}}
	\DeclareGraphicsExtensions{.pdf,.jpeg,.png}
\else
	\usepackage[dvips]{graphicx}
	%\graphicspath{{./figs/}}
	\DeclareGraphicsExtensions{.eps}
\fi

\usepackage[cmex10]{amsmath}
\usepackage[tight,footnotesize]{subfigure}
\usepackage{xcolor}
\usepackage[lined,ruled]{algorithm2e}

\usepackage[latin1]{inputenc}
\usepackage{tikz}
\usetikzlibrary{shapes}
\usetikzlibrary{arrows}

\usepackage[]{algorithm2e}

\newtheorem{property}{Property}
\newtheorem{proposition}{Proposition}
\newtheorem{theorem}{Theorem}
\newtheorem{conjecture}{Conjecture}
\newtheorem{question}{Question}
\newtheorem{definition}{Definition}
\newtheorem{corollary}{Corollary}

\makeatletter
\pgfdeclareshape{datastore}{
\inheritsavedanchors[from=rectangle]
\inheritanchorborder[from=rectangle]
\inheritanchor[from=rectangle]{center}
\inheritanchor[from=rectangle]{base}
\inheritanchor[from=rectangle]{north}
\inheritanchor[from=rectangle]{north east}
\inheritanchor[from=rectangle]{east}
\inheritanchor[from=rectangle]{south east}
\inheritanchor[from=rectangle]{south}
\inheritanchor[from=rectangle]{south west}
\inheritanchor[from=rectangle]{west}
\inheritanchor[from=rectangle]{north west}
\backgroundpath{
    %  store lower right in xa/ya and upper right in xb/yb
\southwest \pgf@xa=\pgf@x \pgf@ya=\pgf@y
\northeast \pgf@xb=\pgf@x \pgf@yb=\pgf@y
\pgfpathmoveto{\pgfpoint{\pgf@xa}{\pgf@ya}}
\pgfpathlineto{\pgfpoint{\pgf@xb}{\pgf@ya}}
\pgfpathmoveto{\pgfpoint{\pgf@xa}{\pgf@yb}}
\pgfpathlineto{\pgfpoint{\pgf@xb}{\pgf@yb}}
 }
}
\makeatother

\newcommand{\riham}[1]{{\color{red}{#1}}}
\newcommand{\james}[1]{{\color{blue}{#1}}}


\begin{document}

\title{CS526 GCRS - Graduate Course Request System - Spring 2018}
\author{
\IEEEauthorblockN{Wael Ayadi, Nick Romano}
\IEEEauthorblockA{Rutgers University\\ Piscataway, NJ, USA\\
Email: wael.ayadi@rutgers.edu, nick.p.romano@gmail.com}
%\and
%\IEEEauthorblockN{Alan Turing}
%\IEEEauthorblockA{Rutgers University\\
% Piscataway, NJ, USA\\
% Email: alan1936@cs.rutgers.edu}
%\and
%%\IEEEauthorblockN{Third group member}
%\IEEEauthorblockA{Rutgers University\\
%\Piscataway, NJ, USA\\
%Email: Third_group_member@scarletmail.rutgers.edu}
}

\maketitle
\begin{abstract}
\textnormal{
We will create a Graduate Course Request System that will allow for easier and faster delivery and approval of graduate course requests for undergraduate students. It will incorporate our multi-purpose Visual Transcript visualization, which provides an interactive interface for extracting useful information from a student's course history. 
}
\end{abstract}
%\onecolumn \maketitle \normalsize \vfill

\IEEEpeerreviewmaketitle
%%%%%%%%%%%%%%%%%%%%%%%%%%%%%%%%%%%%%%%%%%%%%%%%%%%%%%%%%%%%%%%%%%%%%%%%%%%%%%%%%%%%%%%%%%%%%%%%%%%%%%%%%
\section{Project Description}\label{sec:1. Project Description}
%%%%%%%%%%%%%%%%%%%%%%%%%%%%%%%%%%%%%%%%%%%%%%%%%%%%%%%%%%%%%%%%%%%%%%%%%%%%%%%%%%%%%%%%%%%%%%%%%%%%%%%%%
\textnormal{
The Rutgers University Graduate CS Department allows for undergraduate students to register for graduate courses through a special request process. This process, as it is now, involves manually signing a form for every graduate course that the student wishes to register for. This signed form must then be signed by the instructor of the course that is being requested. Afterwards, the form is given to the graduate secretary of the Graduate CS Department. Depending on whether or not the requested course requires a prerequisite override (it often does, considering the student submitting this request is an undergraduate student), the form would then have to be forwarded to the graduate school for approval. This step of the process specifically may take up to a full week to complete! Once the course request is finally approved by the graduate school, the student receives his Special Permission Number (SPN) for the requested course.
}

Not only is this process tedious to describe, but it is tedious in practice as well. The student often has to chase down instructors for their approvals in person, as it is not realistic for them to constantly check their emails for course requests, as well as do the necessary background checks on the student to see if an approval for their request is appropriate. Further complications occur if an instructor is on sabbatical leave, if the graduate secretary is taking a day off, etc.. To make matters worse, students are bound by the add/drop week deadline for course registration at the beginning of every semester, thus making this registration process a timely matter that cannot always be done before the start of the semester. These issues are further compounded by the fact that the Rutgers Graduate CS Department has been growing in terms of student enrollment. The increased frequency of these requests end up delaying this process even further. 

We propose the GCRS - Graduate Course Request System, which is a system that can be linked with Rutgers's Central Authentication System (CAS) to allow for Rutgers faculty to digitally manage and pass along student graduate course requests. This system will use our Visual Transcript visualization, which will allow easy and fast data visualization of a student's course history and related data. Incorporating Visual Transcript into GCRS will give an interactive interface for faculty to quickly see the information they require when it comes to approving these requests. 

%\subsection{Stage1 - The Requirement Gathering Stage. } \label{sec:1	Requirement Gathering Stage. } 
%The goal for this project by the end of the semester is to have one administrative view that incorporates a visual interface for the delivery of user feedback to the 3D-Matching algorithm, and a visual representation of the final matching that is output by the algorithm. 

An example scenario would be as follows:
\\
The CS department administrator has collected the list of TA applications in the CS department and is ready to match them with professors and courses. The administrator logs onto RTAS and runs the 3D-Matching algorithm. The algorithm returns a matching of size n, as well as a visualization of the list of accepted applications. The administrator then notices that Professor A submit a request to have Student A as his TA for Class A. The administrator goes through the preference selection interface and selects the tuple (Student A, Professor A, Course A). This adds the tuple to the algorithm's priority list. He re-runs the algorithm, which returns a new matching with the same size as the old one. The administrator decides to keep it. The administrator inputs a second request: (Student B, Professor B, Course B). The algorithm runs and returns a matching that is smaller than the old matching. The administrator then declines that request and removes tuple (Student B, Professor B, Course B) from the priority list. The administrator then finalizes the final matching and assigns the TAs to their respective professors and courses. 

A tentative timeline for project completion is as follows:

\begin{itemize} 
	\item{\textbf{Feb 19:} A dummy database for use for this project will be generated and prepared for use.} 
	
	\item{\textbf{Mar 6:} The 3D-Matching approximation algorithm will be completed and ready for use.}
	
	\item{\textbf{Mar 20:} A maximal matching visualization prototype will be finished.}
	
	\item{\textbf{Apr 3:} The matching visualization will be completed. The user feedback interactive interface will be completed.}
	
	\item{\textbf{Apr 17:} The administrative system view will be completed.}
	
	\item{\textbf{Apr 24:} The powerpoint presentation will be completed.}
	
	\item{\textbf{May 1:} The project report will be completed.}
	
\end{itemize}

\subsection{Stage1 - The Requirement Gathering Stage. }\label{sec:1 Requirement Gathering Stage. }
%%%%%%%
The goal for this project by the end of the semester is to have one administrative view that incorporates a visual interface for the delivery of user feedback to the 3D-Matching algorithm, and a visual representation of the final matching that is output by the algorithm. 

An example scenario would be as follows:
\\
The CS department administrator has collected the list of TA applications in the CS department and is ready to match them with professors and courses. The administrator logs onto RTAS and runs the 3D-Matching algorithm. The algorithm returns a matching of size n, as well as a visualization of the list of accepted applications. The administrator then notices that Professor A submit a request to have Student A as his TA for Class A. The administrator goes through the preference selection interface and selects the tuple (Student A, Professor A, Course A). This adds the tuple to the algorithm's priority list. He re-runs the algorithm, which returns a new matching with the same size as the old one. The administrator decides to keep it. The administrator inputs a second request: (Student B, Professor B, Course B). The algorithm runs and returns a matching that is smaller than the old matching. The administrator then declines that request and removes tuple (Student B, Professor B, Course B) from the priority list. The administrator then finalizes the final matching and assigns the TAs to their respective professors and courses. 

A tentative timeline for project completion is as follows:

\begin{itemize} 
	\item{\textbf{Feb 19:} A dummy database for use for this project will be generated and prepared for use.} 
	
	\item{\textbf{Mar 6:} The 3D-Matching approximation algorithm will be completed and ready for use.}
	
	\item{\textbf{Mar 20:} A maximal matching visualization prototype will be finished.}
	
	\item{\textbf{Apr 3:} The matching visualization will be completed. The user feedback interactive interface will be completed.}
	
	\item{\textbf{Apr 17:} The administrative system view will be completed.}
	
	\item{\textbf{Apr 24:} The powerpoint presentation will be completed.}
	
	\item{\textbf{May 1:} The project report will be completed.}
	
\end{itemize}


\subsection{Stage2 - The Design Stage. }\label{sec: 2:The Design Stage.}
%%%%%%%%%%%%%%%%%%%%%%%%%%%%%%%%%%%%%%%%%%%%%%%%%%%%%%%%%%%%%%%%%%%%%%%%%%%%%%%%%%%%%%%%%%%%%%%%%%%%%%%%%%
This project will use various technologies for both the front-end and the back-end. For the back end, we plan to use an SQL database to store the data, as well as Java servlets for code execution. For the front-end, we plan on using Angular and Javascript for the layout and D3.js for the Visual Transcript instances. We may use jQuery for SQL query manipulation that will serve as input to the algorithm that we formulate that will retrieve the necessary data and spawn a Visual Transcript instance on each view. 

\textnormal{
Transform the project requirements into a system flow diagram, specifyng the different algorithms, data types and structures required for processing and their associated operations.  
The deliverables for this stage include the system flow diagram containing a graphical representation and  textual descriptions of the corresponding data trasnformations, high level pseudo code of the overall system operation, and overall system time and space complexity.}

%\begin{itemize} 
%\item{ }
%A brief textual description of the overall flow diagram (along with its functional operation in the different user scenarios described in the first stage of the project).
%\item{ }
%A specification of each algorithm and associated data structures together with its entities, attributes, and operations ( include an English description of how they relate to your user scenario(s)).

%\end{itemize}
Please insert your deliverables for Stage2 as follows:
\begin{itemize} 
\item{  Short Textual Project Description. }
Please insert here the flow diagram textual description here together with its overall time and space complexity.
\item{ Flow Diagram. }
Please insert your system Flow Diagram here.
\item{ High Level Pseudo Code System Description. }
Please insert high level pseudo-code describing the major system modules as per your flow diagram.
\item{Algorithms and  Data Structures. }
Please insert a brief description of each major Algorithm and its associated data structures here.
\end{itemize}

\begin{itemize} 
\item{  Flow Diagram Major Constraints.}
Please insert here the integrity constraints:
\begin{itemize} 
\item{ Integrity Constraint. }
Please insert the first integrity constraint in here together with its description and justification. 
\end{itemize}
Please repeat the pattern for each integrity constraint.
\end{itemize}


\subsection{Stage3 - The Implementation Stage. }\label{sec: 3 The Implementation Stage.}
%%%%%%%%%%%%%%%%%%%%%%%%%%%%%%%%%%%%%%%%%%%%%%%%%%%%%%%%%%%%%%%%%%%%%%%%%%%%%%%%%%%%%%%%%
The intended programming languages, programming environments, and technologies that will be used for this project have been listed in Stage 2. The remaining deliverables for this section are a work in progress. 


\subsection{Stage4 -	User Interface. }\label{sec: 4. User Interface.}
%%%%%%%%%%%%%%%%%%%%%%%%%%%%%%%%%%%%%%%%%%%%%%%%%%%%%%%%%%%%%%%%%%%%%%%%%%%%%%%%%%%%%%%%%%%%%%%%%%%%%%%%%%
There will be two main components of the user interface: the requests view (i.e. the main page), and the visual transcript view. In addition there will be a login page for the entire system. The type of user will determine exactly what content and/or interactivity is available in each view. The user type will be assessed via login credentials submitted on the login page. For example, the requests main page will give a student the option to submit a request. This page will also display any open requests and additionally it may show any closed requests that were previously completed. On the opposite side of the pipeline, faculty/admin will have a similar view showing all of the requests that require completion. Both types of users will have access to the visual transcript view which will provide an interactive visualization of the student's course history.This view will allow student users to confirm whether they meet the necessary requirements, and will allow faculty/admin to quickly identify whether a student has the necessary coursework to needed to approve the request. The faculty/admin view will contain "Approve" and "Deny" buttons which will either continue or terminate the request, respectively. Each user view is described in further detail herein. As the login page will not be a main component of the user interface, it is assumed the user has already logged in.

\begin{figure}
	\includegraphics[width=\linewidth]{login_page.png}
	\caption{Initial Login View}
	\label{fig:fig1}
\end{figure}

For student users, the requests main page will contain a link that will take the student to their visual transcript view, and buttons to submit a new request. Any open requests will be visible in a list with options to view its status. Upon click of an open request, a window will open showing details on the status of that request. Alternatively, the status of an open request may be shown upon hover. Any closed requests (i.e. requests that were previously completed will be displayed at the bottom of the page). A wireframe representation for the requests view is given Figure 2.

\begin{figure}
	\includegraphics[width=\linewidth]{requests_main.png}
	\caption{Main Requests View}
	\label{fig:fig1}
\end{figure}

For faculty/admin users, the requests main page will appear much like the students requests page however the options available to the user will be different. In this view, faculty or admin will not require the option to submit a request. Instead all open requests (submitted by the students) will be displayed. Similar to the student view of the requests main page, the faculty/admin view will allow the user to view the status of any request and previously completed requests will be shown at the bottom of the page. The faculty/admin user will have the options to approve or deny the request (or multiple requests) from this page. The faculty/admin user will also have the option to access the students virtual transcript in order to view the students course history.

\begin{figure}
	\includegraphics[width=\linewidth]{virtual_transcript.png}
	\caption{Virtual Transcript View}
	\label{fig:fig1}
\end{figure}

The purpose of this virtual transcript is to provide a visual and interactive aspect to the traditional course list and grade type transcript. This visual transcript will allow all users to quickly interpret a student's academic history. The data to be visualized in this virtual transcript will be the complete course history of the particular student as well as a visual representation of the students strengths (or weaknesses) It is imagined that the students course history will be displayed in sequential order through an adapted bar chart where the bars are located over each course and the scale of the vertical axis represents the grade the student received for the course. A color dimension will be added to represent the course level (e.g. 100, 200, 300-level courses). Additionally the students strengths and/or weaknesses in certain course concentrations (e.g. machine learning, security, data science, etc.) will be displayed via a set radial progress bars.


\section{Project Highlights.}\label{sec:7. Project Highlights.}
%%%%%%%%%%%%%%%%%%%%%%%%%%%%%%%%%%%%%%%%%%%%%%%%%%%%%%%%%%%%%%%%%%%%%%%%%%%%%%%%%%%%%%%%%%%%%%%%%%%%%%%%%%
\input{
.tex}


\bibliographystyle{IEEEtran}
%\bibliography{IEEEabrv,bib_queyroi_abello2013}
%\bibliography{bib_queyroi_abello2013}

\end{document}


